% !TeX document-id = {1a0818d7-2ba8-473c-9e6c-844111d97c08}
% !TEX TS-program = xeLaTeX

\documentclass[12pt]{article}
\usepackage{ifxetex,ifluatex}
\usepackage{etoolbox}
\usepackage[svgnames]{xcolor}

\usepackage{tikz}

\usepackage{framed}

% conditional for xetex or luatex
\newif\ifxetexorluatex
\ifxetex
  \xetexorluatextrue
\else
  \ifluatex
    \xetexorluatextrue
  \else
    \xetexorluatexfalse
  \fi
\fi
%
\ifxetexorluatex%
  \usepackage{fontspec}
  \usepackage{libertine} % or use \setmainfont to choose any font on your system
  \newfontfamily\quotefont[Ligatures=TeX]{Linux Libertine O} % selects Libertine as the quote font
\else
  \usepackage[utf8]{inputenc}
  \usepackage[T1]{fontenc}
  \usepackage{libertine} % or any other font package
  \newcommand*\quotefont{\fontfamily{LinuxLibertineT-LF}} % selects Libertine as the quote font
\fi

\newcommand*\quotesize{60} % if quote size changes, need a way to make shifts relative
% Make commands for the quotes
\newcommand*{\openquote}
   {\tikz[remember picture,overlay,xshift=-4ex,yshift=-2.5ex]
   \node (OQ) {\quotefont\fontsize{\quotesize}{\quotesize}\selectfont``};\kern0pt}

\newcommand*{\closequote}[1]
  {\tikz[remember picture,overlay,xshift=4ex,yshift={#1}]
   \node (CQ) {\quotefont\fontsize{\quotesize}{\quotesize}\selectfont''};}

% select a colour for the shading
\colorlet{shadecolor}{pink}

\newcommand*\shadedauthorformat{\emph} % define format for the author argument

% Now a command to allow left, right and centre alignment of the author
\newcommand*\authoralign[1]{%
  \if#1l
    \def\authorfill{}\def\quotefill{\hfill}
  \else
    \if#1r
      \def\authorfill{\hfill}\def\quotefill{}
    \else
      \if#1c
        \gdef\authorfill{\hfill}\def\quotefill{\hfill}
      \else\typeout{Invalid option}
      \fi
    \fi
  \fi}
% wrap everything in its own environment which takes one argument (author) and one optional argument
% specifying the alignment [l, r or c]
%
\newenvironment{shadequote}[2][l]%
{\authoralign{#1}
\ifblank{#2}
   {\def\shadequoteauthor{}\def\yshift{-2ex}\def\quotefill{\hfill}}
   {\def\shadequoteauthor{\par\authorfill\shadedauthorformat{#2}}\def\yshift{2ex}}
\begin{snugshade}\begin{quote}\openquote}
{\shadequoteauthor\quotefill\closequote{\yshift}\end{quote}\end{snugshade}}

\begin{document}

\begin{shadequote}[r]{\textit\textit{Karsten Danzmann, Experimentalphysik}}
Dieses Rosa macht mich ganz wuschig
\end{shadequote}
\begin{shadequote}[r]{\textit\textit{M. Aulwurf, Experimentalphysik Tutorium}}
	Hat das Eisenatom dann nur ein Proton?
\end{shadequote}
\begin{shadequote}[r]{\textit\textit{Alex, Experimentalphysik Tutorium}}
	Nabla \times BlubBlub
\end{shadequote}
\begin{shadequote}[r]{\textit\textit{Alex, Experimentalphysik Tutorium}}
	Wenn man sonst nur in natürlichen Einheiten rechnet und sich denkt: so what ?!?
\end{shadequote}
\begin{shadequote}[r]{\textit\textit{Michelle Heurs, Experimentalphysik}}
	Ich muss mich mal einweisen lassen….
\end{shadequote}
\begin{shadequote}[r]{\textit\textit{Michelle Heurs, Experimentalphysik}}
	Ich löte auch gerne!
\end{shadequote}
\begin{shadequote}[r]{\textit\textit{Michelle Heurs, Experimentalphysik}}
	Herr Schlenk darf die Spule dann einführen.
\end{shadequote}
\begin{shadequote}[r]{\textit\textit{Michelle Heurs, Experimentalphysik}}
	Cooler Versuch, das hat aber nicht nur was mit dem Stickstoff zu tun.
\end{shadequote}
\begin{shadequote}[r]{\textit\textit{Alex, Experimentalphysik Tutorium}}
	...über den Oschi integrieren...
\end{shadequote}
\begin{shadequote}[r]{\textit\textit{Reinhart Werner,  TED}}
	...die Additivität der Addition...
\end{shadequote}
\begin{shadequote}[r]{\textit\textit{Reinhart Werner,  TED}}
	Ihr wollt ein Skript, kriegt ihr aber nicht!
\end{shadequote}
\begin{shadequote}[r]{\textit\textit{Emil Wiedemann,  Analysis II}}
	...im 19. Jahrhundert war das der heiße Scheiß!
\end{shadequote}
\begin{shadequote}[r]{\textit\textit{Ghislain Fourier,  Lineare Algebra I}}
	Sie sind, es sei denn sie kommen aus der Eiffel, \underline{nicht} mit ihrem Ehepartner verwandt.
\end{shadequote}
\begin{shadequote}[r]{\textit\textit{Ghislain Fourier,  unbekannt}}
	Solange die dicke Frau noch singt, ist die Oper nicht vorbei.
\end{shadequote}
\begin{shadequote}[r]{\textit\textit{Karsten Danzmann,  Experimentalphysik I}}
	...also die Farbe [pink] macht mich ganz wuschig.
\end{shadequote}
\begin{shadequote}[r]{\textit\textit{Karsten Danzmann,  Experimentalphysik I}}
	Wir haben ausgerechnet, was wir ausgerechnet haben.
\end{shadequote}
\begin{shadequote}[r]{\textit\textit{Karsten Danzmann,  Experimentalphysik I}}
	Die Pupillen weiten sich...jaa, das kann man auch anders erreich, ich weiß...
\end{shadequote}
\begin{shadequote}[r]{\textit\textit{Karsten Danzmann,  Experimentalphysik I}}
	Eine Kilowattstunde, das ist schon ein ziemlicher Johnny!
\end{shadequote}
\begin{shadequote}[r]{\textit\textit{Matthias Schlenk,  Experimentalphysik I}}
	Im Dunkeln hat der Blitz Angst!
\end{shadequote}
\begin{shadequote}[r]{\textit\textit{Matthias Schlenk,  Experimentalphysik I}}
	...das ist kein blau, das weiß ich ganz genau hoho….
\end{shadequote}
\begin{shadequote}[r]{\textit\textit{Matthias Schlenk&Karsten Danzmann,  Experimentalphysik I}}
	Schlenk: Wir brauchen Katzenfell!\\
	Danzmann:Man kann auch Katzen nehmen.\\
	Schlenkt: Aber die katze vorher tot machen.\\
	Danzmann: man kann sonst auch ein Stück rausschneiden. Aber lieber eine tote Katze nehmen, lebendige sind meist nass.
\end{shadequote}
\end{document}